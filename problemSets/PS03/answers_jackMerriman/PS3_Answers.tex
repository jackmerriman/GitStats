\documentclass[12pt,letterpaper]{article}
\usepackage{graphicx,textcomp}
\usepackage{natbib}
\usepackage{setspace}
\usepackage{fullpage}
\usepackage{color}
\usepackage[reqno]{amsmath}
\usepackage{amsthm}
\usepackage{fancyvrb}
\usepackage{amssymb,enumerate}
\usepackage[all]{xy}
\usepackage{endnotes}
\usepackage{lscape}
\newtheorem{com}{Comment}
\usepackage{float}
\usepackage{hyperref}
\newtheorem{lem} {Lemma}
\newtheorem{prop}{Proposition}
\newtheorem{thm}{Theorem}
\newtheorem{defn}{Definition}
\newtheorem{cor}{Corollary}
\newtheorem{obs}{Observation}
\usepackage[compact]{titlesec}
\usepackage{dcolumn}
\usepackage{tikz}
\usetikzlibrary{arrows}
\usepackage{multirow}
\usepackage{xcolor}
\newcolumntype{.}{D{.}{.}{-1}}
\newcolumntype{d}[1]{D{.}{.}{#1}}
\definecolor{light-gray}{gray}{0.65}
\usepackage{url}
\usepackage{listings}
\usepackage{color}

\definecolor{codegreen}{rgb}{0,0.6,0}
\definecolor{codegray}{rgb}{0.5,0.5,0.5}
\definecolor{codepurple}{rgb}{0.58,0,0.82}
\definecolor{backcolour}{rgb}{0.95,0.95,0.92}

\lstdefinestyle{mystyle}{
	backgroundcolor=\color{backcolour},   
	commentstyle=\color{codegreen},
	keywordstyle=\color{magenta},
	numberstyle=\tiny\color{codegray},
	stringstyle=\color{codepurple},
	basicstyle=\footnotesize,
	breakatwhitespace=false,         
	breaklines=true,                 
	captionpos=b,                    
	keepspaces=true,                 
	numbers=left,                    
	numbersep=5pt,                  
	showspaces=false,                
	showstringspaces=false,
	showtabs=false,                  
	tabsize=2
}
\lstset{style=mystyle}
\newcommand{\Sref}[1]{Section~\ref{#1}}
\newtheorem{hyp}{Hypothesis}

\title{Problem Set 3}
\date{Submitted: March 26, 2023}
\author{Jack Merriman}


\begin{document}
	\maketitle
\section*{Question 1} 

\noindent \textbf{Part 1:}\\
\lstinputlisting[language = R, firstline=13, lastline=36]{PS03.R}
\newpage
% Table created by stargazer v.5.2.3 by Marek Hlavac, Social Policy Institute. E-mail: marek.hlavac at gmail.com
% Date and time: Sun, Mar 26, 2023 - 15:40:10
\begin{table}[!htbp] \centering   \caption{}   \label{}
	\begin{tabular}{@{\extracolsep{5pt}}lcc} \\[-1.8ex]\hline \hline \\[-1.8ex]  & \multicolumn{2}{c}{\textit{Dependent variable:}} \\ \cline{2-3} \\[-1.8ex] & negative & positive \\ \\[-1.8ex] & (1) & (2)\\ \hline \\[-1.8ex]  OIL & 4.784 & 4.576 \\   & (6.885) & (6.885) \\   & & \\  REG & 1.379$^{*}$ & 1.769$^{**}$ \\   & (0.769) & (0.767) \\   & & \\  Constant & 3.805$^{***}$ & 4.534$^{***}$ \\   & (0.271) & (0.269) \\   & & \\ \hline \\[-1.8ex] Akaike Inf. Crit. & 4,690.770 & 4,690.770 \\ \hline \hline \\[-1.8ex] \textit{Note:}  & \multicolumn{2}{r}{$^{*}$p$<$0.1; $^{**}$p$<$0.05; $^{***}$p$<$0.01} \\
\end{tabular}
\end{table} 

\begin{BVerbatim}
	
	
Predicted Values
                       no change  negative  positive
<50%, non-democracy 7.191671e-03 0.3232070 0.6696013
>50%, non-democracy 6.934296e-05 0.3726529 0.6272778
<50%, democracy     1.378186e-03 0.2460217 0.7526001
>50%, democracy     1.344048e-05 0.2869004 0.7130862
\end{BVerbatim}
\clearpage

\noindent \textit{OIL:}\\
\noindent A country having an oil to total exports ratio $>50\%$ will see an average increase of $4.784$ in the log odds of having negative economic growth vs. no economic growth, and an average increase of $4.576$ in the log odds of having positive economic growth vs. no economic growth compared to a  country with an oil to exports ratio $<50\%$, holding all other variables constant.\\

\noindent \textit{REG:}\\
\noindent A democratic country will see an average increase of $1.379$ in the log odds of having negative economic growth vs. no economic growth, and an average increase of $1.769$ in the log odds of having positive economic growth vs. no economic growth compared to a non-democratic country, holding all other variables constant.\\

\noindent \textit{Constant:}\\
\noindent A non-democratic country with an oil to total exports ratio $<50\%$ has an average log odds of $3.805$ of having negative economic growth vs. no economic growth, and an average log odds of $4.534$ of having positive economic growth vs. no economic growth.\\

\noindent \textbf{Part 2:}\\

% Table created by stargazer v.5.2.3 by Marek Hlavac, Social Policy Institute. E-mail: marek.hlavac at gmail.com
% Date and time: Sun, Mar 26, 2023 - 15:47:29
\begin{table}[!htbp] \centering   \caption{}   \label{} \begin{tabular}{@{\extracolsep{5pt}}lc} \\[-1.8ex]\hline \hline \\[-1.8ex]  & \multicolumn{1}{c}{\textit{Dependent variable:}} \\ \cline{2-2} \\[-1.8ex] & oGDPWdiff \\ \hline \\[-1.8ex]  OIL & $-$0.199$^{*}$ \\   & (0.116) \\   & \\  REG & 0.398$^{***}$ \\   & (0.075) \\   & \\ \hline \\[-1.8ex] Observations & 3,721 \\ \hline \hline \\[-1.8ex] \textit{Note:}  & \multicolumn{1}{r}{$^{*}$p$<$0.1; $^{**}$p$<$0.05; $^{***}$p$<$0.01} \\ \end{tabular} \end{table} 
\clearpage
\lstinputlisting[language = R, firstline = 54, lastline=64]{PS03.R}
\begin{BVerbatim}
	             negative   no change  positive
<50%, non-democracy 0.3249362 0.004555476 0.6705083
>50%, non-democracy 0.3699431 0.004836151 0.6252207
<50%, democracy     0.2442235 0.003839727 0.7519368
>50%, democracy     0.2827332 0.004215307 0.7130515
\end{BVerbatim}

\vspace{0.75cm}
\noindent \textit{OIL:}\\
\noindent A country having an oil to total exports ratio $>50\%$ will see an average decrease of $0.199$ in the log odds of moving up an ordinal category from negative compared to a country with an oil to exports ratio $<50\%$, holding all other variables constant.\\

\noindent \textit{REG:}\\
\noindent A democratic country will see an average increase of $0.398$ in the log odds of moving up an ordinal category from negative compared to a non-democratic country, holding all other variables constant.\\

\newpage
\section*{Question 2}
\vspace{.25cm}
\noindent\textit{(a)}\\

\lstinputlisting[language = R, firstline = 70, lastline=75]{PS03.R}

% Table created by stargazer v.5.2.3 by Marek Hlavac, Social Policy Institute. E-mail: marek.hlavac at gmail.com
% Date and time: Sun, Mar 26, 2023 - 15:49:06
\begin{table}[!htbp] \centering   \caption{}   \label{} \begin{tabular}{@{\extracolsep{5pt}}lc} \\[-1.8ex]\hline \hline \\[-1.8ex]  & \multicolumn{1}{c}{\textit{Dependent variable:}} \\ \cline{2-2} \\[-1.8ex] & PAN.visits.06 \\ \hline \\[-1.8ex]  competitive.district & $-$0.081 \\   & (0.171) \\   & \\  marginality.06 & $-$2.080$^{***}$ \\   & (0.117) \\   & \\  PAN.governor.06 & $-$0.312$^{*}$ \\   & (0.167) \\   & \\  Constant & $-$3.810$^{***}$ \\   & (0.222) \\   & \\ \hline \\[-1.8ex] Observations & 2,407 \\ Log Likelihood & $-$645.606 \\ Akaike Inf. Crit. & 1,299.213 \\ \hline \hline \\[-1.8ex] \textit{Note:}  & \multicolumn{1}{r}{$^{*}$p$<$0.1; $^{**}$p$<$0.05; $^{***}$p$<$0.01} \\ \end{tabular} \end{table} 

\noindent The coefficient associated with the competitive district is very small at $-0.081$, the \texttt{summary()} function in R outputs a Z test statistic for the competitive district coefficient of $-0.477$ which equates to a p-value of $0.6336$, which is not statistically significant at any reasonable level of confidence, therefore we can conclude that there is no evidence that PAN presidential candidates visit swing districts more frequently.\\

\newpage
\noindent\textit{(b)}\\
\noindent We first exponentiate our coefficients to interpret them:
\lstinputlisting[language = R, firstline =78, lastline=78]{PS03.R}
\begin{BVerbatim}
(Intercept) competitive.district       marginality.06      PAN.governor.06
 0.02214298           0.92186932           0.12491227           0.73228985 
\end{BVerbatim}

\vspace{0.75cm}
\noindent \texttt{marginality.06:}\\
\noindent An increase of one in the marginality score (poverty measure) in a district is associated with a decrease in the expected number of visits by the PAN candidate to the district by a multiplicative factor of $0.124$, holding all other variables constant.\\

\noindent \texttt{PAN.governor.06:}\\
\noindent Districts with PAN governors are associated with an average decrease in the expected number of visits by the PAN candidate to the district by a multiplicative factor of $0.73$, relative to districts without a PAN governor, holding all other variables constant.\\

\noindent\textit{(c)}
\lstinputlisting[language = R, firstline=84,lastline=86]{PS03.R}
$\lambda = 0.002$\\
\noindent There is a mean expected visits number of $0.002$ which implies it is unlikely that a PAN candidate would visit this district.


\end{document}
